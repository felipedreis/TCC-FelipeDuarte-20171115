% -----------------------------------------------------------------------------
%   Arquivo: ./02-elementos-textuais/introducao.tex
% -----------------------------------------------------------------------------

\begin{comment}
 contexto: simulação de sistemas ecológicos utilizando MAS
 motivação: simular sistemas complexos e de larga escala com plausabilidade biológica
 justificativa: assincronia intrínseca deste tipo de modelo é incompatível com os modelos de programação utilizando memória compartilhada
 objetivo: implementar um simulador escalável utilizando o modelo de atores, bem como um sistema nervoso assíncrono
 metodologia: usar o toolkit akka implementado na linguagem Java para construir o mundo e a criatura artificial, testar os circuitos de estimulação da dinâmica interna, verificar e validar o modela via simulação de forrageamento
 resultado: Escalabilidade. Espera-se fazer as simulações em larga escala e com os resultados compatíveis com a literatura
\end{comment}

\chapter{Introdução}
\label{chap:introducao}

% Contexto: Sistemas inteligentes, sistemas multi-agentes
Dentre as abordagens propostas na computação para a construção de sistemas inteligentes,
tiveram maior destaque os sistemas especialistas, a lógica fuzzy, as redes neurais artificiais,
sistemas bio-inspirados, e os sistemas multi-agentes (MAS), sendo essas não excludentes. MAS são aplicados na modelagem e simulação de sistemas complexos, \textit{e.g.} sistemas ecológicos, sociais e econômicos, a fim de estudar a emergência de fenômenos na dinâmica do sistema \cite{Niazi2011}. Essa abordagem tem sido aplicada também nos mais diversos setores da indústria e economia com sucesso \cite{Muller2014}.

% Motivação: simular sistemas com plausabilidade biológica em larga escala
Segundo \citeonline{Niazi2011} MAS são frequentemente empregados em modelagem de sistemas ecológicos e biológicos, em especial nas aplicações de vida artificial e cognição. Estudar a emergência de fenômenos como adaptação, aprendizagem, linguagem e consciência é um desafio nessa área de pesquisa \cite{Bedau2000}. \citeonline{Duch2008}  classifica várias arquiteturas cognitivas  baseando-se na abordagem usada na construção de cada uma. Essas classes são: simbólicas, emergentes e híbridas. Arquiteturas simbólicas são baseadas em um processador de símbolos de alto nível, e partem de uma perspectiva \textit{top-down}. Arquiteturas emergentes partem de um fundamento conexionista, baseado em redes neurais e abordam a cognição da perspectiva \textit{bottom-up}. A categoria de arquiteturas híbridas combina ambas as abordagens para produzir um modelo que exiba um comportamento plausível com o fenômeno biológico da cognição, em especial, o que se observa nos seres humanos. Nesta última categoria se enquadram algumas arquiteturas cognitivas multi-agentes populares, como por exemplo, LIDA e CLARION \cite{Franklin2006, Sun2001}. Um outro trabalho baseado na teoria incorporada da cognição é a arquitetura Artífice, apresentada por \citeonline{Campos2015}. Essas arquiteturas em geral são complexas, compostas de vários componentes que interagem entre si, podendo eles serem concorrentes e não-determinísticos, que é o caso da última citada.  

Um modelo fiel de um processo cognitivo é naturalmente complexo e a implementação de uma arquitetura baseada em tal modelo certamente terá um alto custo computacional. \citeonline{Prasad1996} mostraram que ao aumentar a complexidade de um sistema multi-agente acarreta em impactos na sua performance, escalabilidade e estabilidade. O autor também argumenta que a escalabilidade não é um atributo do modelo, e sim, da implementação, e está relacionada principalmente com as tecnologias adotadas. Portanto, para estudar sistemas complexos, principalmente os que modelam fenômenos biológicos, executando simulações em larga escala, é necessário escolher um ferramental tecnológico adequado.

%Justificativa: assincronia intrínseca deste tipo de modelo é incompatível com os modelos de programação utilizando memória compartilhada
Quando a área de estudo envolve sistemas dinâmicos, ecológicos ou qualquer outro processo de origem biológica, como é o caso de processos cognitivos de criaturas artificiais - contexto no qual este trabalho se insere, há dois aspectos que parecem relevantes e devem ser considerados: tais processos são inerentemente não-determinísticos \cite{Symmons2016} e assíncronos \cite{Fisher2008, Strombom2017, Wang2016}. Consequente, um modelo matemático e/ou computacional que vise manter algum grau de plausibilidade biológica para com o fenômeno modelado, deverá c    onsequentemente manter pelo menos uma destas duas características, quais sejam o não-determinismo e a assincronia. Todavia, no que concerne ao aspecto computacional, a larga maioria dos modelos apresentados na literatura recorre ao uso de \textit{threads} e controle de concorrência síncronos. 

O mecanismo de concorrência por \textit{threads} surgiu na década de 70 com o objetivo de produzir sistemas mais confiáveis \cite{Hansen2013}. Uma \textit{thread} é um trecho de código de um programa que pode executar simultaneamente à outros. Quando duas \textit{threads} tentam acessar a mesma posição de memória ao mesmo tempo isso causa uma condição de corrida, produzindo um estado inválido da memória. Para controlar os acessos concorrentes usa-se um mecanismo chamado semáforo binário ou \textit{mutex}, que bloqueia o acesso de uma \textit{thread} enquanto outra está fazendo o acesso. Esse tipo de solução bloqueante não é escalável uma vez que a medida que o número de \textit{threads} cresce, a concorrência pelos recursos compartilhados aumenta, fazendo que o sistema execute praticamente de forma sequencial, o que diminui o desempenho computacional.  

Atualmente outros modelos de concorrência tem surgido, principalmente com a necessidade de se produzir sistemas mais escaláveis. Como por exemplo o modelo de atores \cite{Tasharofi2013, Hewitt2010,  Haller2012}, proposto inicialmente por \citeonline{Hewitt1973} como um formalismo para construção de softwares inteligentes. Um ator é uma primitiva universal da computação digital, não compartilha estado com outros atores e pode se comunicar com estes via troca de mensagens assíncrona. Ao receber uma mensagem um ator pode: alterar seu estado interno, criar um número finito de atores, ou enviar um número finito de mensagens a outro ator. A assincronia intrínseca do modelo de atores permite produzir sistemas mais escaláveis, aproveitando ao máximo vários núcleos de um mesmo computador, bem como vários computadores em um \textit{cluster}. Essa característica do modelo é bem compatível com os sistemas multi-agentes que modelam processos cognitivos, onde os eventos também podem ocorrer de maneira não simultânea.

Apesar de tanto o modelo de atores quando o de \textit{threads} terem surgido praticamente na mesma época, o primeiro começou a ganhar popularidade recentemente com a popularização das aplicações \textit{web} e do \textit{cloud computting}. Uma das implementações mais robustas deste modelo atualmente é o Akka, inicialmente feito em linguagem Scala e, posteriormente na linguagem Java. 

% Objetivo: implementar um simulador escalável utilizando o modelo de atores, bem como um sistema nervoso assíncrono
Tendo em vista a existência de trabalhos na área de sistemas multi-agentes que necessitam de escalabilidade, construídos sobre modelos de concorrência que dificultam o crescimento desses sistemas, e não obstante, existem modelos alternativos de concorrência que favorecem a escalabilidade, cujas características se adaptam bem à simulação de sistemas cognitivos; o que se propõe neste trabalho é utilizar o modelo de atores para construir um simulador de vida artificial, distribuído e assíncrono,  dotado de um sistema nervoso artificial, que seja escalável.  Este simulador será baseado na arquitetura Artífice, citada anteriormente. Posto este objetivo geral, os objetivos específicos do trabalho são:

\begin{enumerate}
    \item Compreender o modelo de atores aplicado a construção de sistemas multi-agentes
    \item Estudar e compreender a arquitetura Artífice 
    \item Apresentar uma nova versão da arquitetura, baseando-se no modelo de atores
    \item Apresentar um modelo de simulação em ambiente distribuído
    \item Implementar o modelo proposto utilizando o \textit{toolkit} Akka
    \item Executar simulações da arquitetura proposta, extraindo dados da execução, para compará-lo com a literatura do projeto Artífice
\end{enumerate}

Feita esta introdução, o trabalho está organizado da seguinte maneira: o \autoref{chap:fundamentacaoTeorica} introduz o modelo de concorrência por estado compartilhado utilizando \textit{threads} e como surgem os principais problemas de sincronização, apresenta o modelo de atores, uma de suas implementações, o \textit{toolkit}  Akka e como ele resolve esses problemas. O \autoref{chap:trabRelac} apresenta uma breve descrição de algumas arquiteturas cognitivas, em destaque a arquitetura Artífice, e como ela se diferencia dos trabalhos na literatura. O \autoref{chap:metodologia} apresenta o a metodologia, bem como o desenvolvimento do trabalho, e as principais escolhas de projeto. Um conjunto de experimentos preliminares e seus resultados é apresentado no \autoref{chap:resultados}  e por fim o  \autoref{chap:conclusao} faz as considerações finais.

