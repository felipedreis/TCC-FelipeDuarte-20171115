% -----------------------------------------------------------------------------
%   Arquivo: ./02-elementos-textuais/fundamentacaoTeorica.tex
% -----------------------------------------------------------------------------

\chapter{Fundamentação Teórica}
\label{chap:fundamentacaoTeorica}

Neste capítulo serão apresentados conceitos pertinentes para a compreensão do presente trabalho. A \autoref{sec:concorrencia} apresenta os fundamentos de controle de concorrência usando o modelo de \textit{threads}. A \autoref{sec:atores} apresenta o modelo de atores e uma de suas implementações, provendo exemplos que explicam como são eliminados os problemas existentes no clássico de concorrência.

\section{Concorrência usando threads}
\label{sec:concorrencia}

Programação concorrente surge na década de 70 com a necessidade de sistemas operacionais executarem vários processos ao mesmo tempo, e seu desenvolvimento ao longo dos anos teve o objetivo de produzir sistemas mais confiáveis \cite{Hansen2013}. Com ele muitos problemas de pesquisa surgiram, foram abordados e solucionados. Atualmente a maioria das linguagens de programação tem mecanismos consolidados que partiram desse desenvolvimento.

Assim como em um sistema operacional em que muitos processos podem executar concorrentemente, dentro de um mesmo processo várias linhas de execução, doravante \textit{threads}, podem executar ao mesmo tempo. O uso de programação concorrente é fundamental atualmente para fazer uso das arquiteturas \textit{multi-core} modernas. A diferença fundamental entre \textit{threads} é que as primeiras compartilham o mesmo espaço de endereçamento, \textit{i.e.} elas tem acesso às mesmas variáveis e podem executar operações de leitura e escrita nelas simultaneamente. O acesso concorrente a posições na memória compartilhadas sem o devido  controle pode causar problemas a execução de programas \textit{multi-thread}, uma vez que o escalonamento das \textit{threads} é não-determinístico. Os principais problemas que tem de ser controlados são \textit{deadlocks}, \textit{starvation}, e condições de corrida.

Uma \textbf{condição de corrida} acontece quando duas threads entram em uma \textbf{seção crítica} e executam ao mesmo tempo uma operação em memória produzindo um estado inválido. Uma \textbf{seção crítica} é um trecho de código em que a execução de mais de uma \textit{thread} não deve ser permitida \cite[p. 206]{Silberschatz2012}. O \autoref{lst:corrida} mostra um exemplo de código em C que pode exibir um problema de condição de corrida. 

\begin{lstlisting}[language=c,label={lst:corrida},caption={Exemplo de programa \textit{multithread} em C onde ocorre uma condição de corrida}]
#include <pthread.h>
#include <stdio.h>

int cont;

void* inc(void* n) {
    cont = cont + 1;
    return NULL;
}

int main() {
    pthread_t threads[3];
    cont = 0;
    for (int i = 0; i < 3; ++i)
        pthread_create(&threads[i], NULL, &inc, NULL);
    for (int  i = 0; i < 3; ++i)
        pthread_join(threads[i], NULL);
        
    printf("%d\n", cont);
    return 0;
}
\end{lstlisting}

O programa cria três \textit{threads} na linha quinze que executam a função que está definida na linha 6. Essa função simplesmente incrementa o contador global \textbf{cont}. O programa espera todas as \textit{threads} terminarem na linha 17 e exibe o resultado na linha 19. A depender da ordem que os comandos são executados, o resultado final do contador pode não ser igual ao número de \textit{threads} que executaram. A lista abaixo exibe uma ordem das \textit{threads} executadas para \autoref{lst:corrida} que produz o valor final 2 enquanto o esperado seria 3:

\begin{enumerate}
 \item A \textit{thread} 1 lê o valor 0 da variável cont da memória principal
 \item A \textit{thread} 2 lê o valor 0 da variável cont da memória principal
 \item Ambas as \textit{threads} 1 e 2 executam o incremento produzindo o valor 1
 \item A \textit{thread} 1 escreve 1 na variável cont
 \item A \textit{thread} 2 escreve 1 na variável cont
 \item A \textit{thread} 3 executa produzindo o valor 2 na variável cont
\end{enumerate}

Esse problema de ordenação dos comandos pode ser resolvido com uma primitiva de sincronização chamada semáforo binário, ou \textit{mutex}. Essa estrutura controla o acesso à seção crítica, permitindo apenas uma \textit{thread} acessá-la por vez. Se uma segunda \textit{thread} tenta fazer o acesso enquanto primeira está executando a seção crítica, aquela é posta em estado de espera até que o \textit{mutex} seja liberado. 

Quando uma \textit{thread} entra na seção crítica ela deve garantidamente deixar a seção crítica em algum momento, caso contrário as outras \textit{threads} que tentarem acessar a região podem ficar em \textit{\textbf{deadlock}}, \textit{i.e.}, esperando por um recurso que jamais será liberado \cite[p. 217]{Silberschatz2012}. Esse problema pode acontecer quando duas \textit{threads} precisam bloquear dois recursos A e B ao mesmo tempo para poderem executar, mas uma \textit{thread} bloqueou o recurso A e outra bloqueou o recurso B. Este problema pode ser resolvido checando antes, ao conseguir bloquear o primeiro recurso, se o segundo está liberado. Se não estiver, desbloqueia-se o primeiro e entra-se em estado de espera até que ambos sejam liberados.

Ainda segundo \citeonline{Silberschatz2012}, \textbf{\textit{starvation}} é um problema semelhante à \textit{deadlocks} no sentido de que uma \textit{thread} fica esperando por tempo indeterminado por um recurso por nunca ser escalonada. Esse problema é mais raro mas pode acontecer quando o escalonamento é prioritário e uma das \textit{threads} sempre perde prioridade para as demais. 

Como visto, o modelo de programação concorrente usando estado compartilhado pode ser problemático se o controle de concorrência não for feito adequadamente utilizando as primitivas de sincronização. Ainda assim, depurar esse tipo de problema pode ser trabalhoso pois o escalonador das \textit{threads} é geralmente não-determinístico e seria necessário reproduzir o mesmo estado da memória e a mesma ordem de execução dos comandos para produzir o erro esperado. 

O modelo que utiliza compartilhamento de memória é inadequando quando se pensa em sistemas escaláveis, tanto horizontal quanto verticalmente. Por \textbf{escalabilidade vertical} se entende a capacidade de um sistema de utilizar eficientemente todos os recursos de uma única máquina. Em outras palavras, quanto melhor a configuração da máquina, melhor o desempenho do sistema. Por \textbf{escalabilidade horizontal} se entende a capacidade de um sistema de aproveitar ao máximo a quantidade de máquinas disponíveis na rede \cite{Abbadi2011}. 

A escalabilidade vertical do sistemas em memória compartilhada fica comprometida porque, ainda que uma máquina tenha muitos processadores, as \textit{threads} competem  pelos mesmos recursos e bloqueiam a execução uma da outra, impedindo o estado do programa de avançar. O modelo de \textit{threads} também não oferece nenhum mecanismo nativo para que uma \textit{thread} se comunique com uma outra que esteja em outra máquina; para que isso aconteça é necessário implementar a comunicação usando \textit{sockets} ou invocação remota de método (RMI), o que dificulta o projeto de um sistema distribuído e escalável baseado em \textit{threads}. 

Eliminando a necessidade de compartilhamento de estado é possível chegar a um novo modelo de concorrência mais escalável que funciona somente por troca de mensagens, que é objeto da próxima seção.

\section{Modelo de atores}
\label{sec:atores}

O modelo de atores é uma alternativa ao modelo clássico de programação concorrente \cite{Agha1985}. Foi proposto por \citeonline{Hewitt1973} como um formalismo para modelar softwares inteligentes, baseado em uma única unidade computacional. Essa unidade, os \textbf{atores}, são uma primitiva universal da computação digital que se comunicam por troca de mensagens. 

Um ator é uma entidade computacional isolada e deve ser capaz de criar outros atores, mudar seu estado interno e enviar mensagens a outros atores \cite{Hewitt2010}. No que diz respeito às mensagens, não há nenhuma restrição sobre seu tamanho, conteúdo ou conjunto de símbolos que a compõe. A única restrição sobre elas é que sejam um conjunto de dados tipado. O ator é isolado no sentido de que não compartilha estado com outros atores e toda comunicação é feita, a principio, por troca de mensagens assíncrona. Essa característica intrínseca do modelo de atores implica na não existência de condições de corrida, e por isso, não existem semáforos nem primitivas de sincronização no modelo.

Os modelos clássicos de computação, como o de Turing e o \textit{lambda-calculus} são casos particulares do modelo de atores\cite{Hewitt2013}. Quando o modelo de atores foi proposto, algumas linguagens implementaram seus princípios como a PLANNER \cite{Hewitt1973}. Atualmente existem implementações em linguagens mais modernas como Erlang, Scala, Java e C\# e que tem tido, inclusive, uso comercial.

Existem outros modelos de programação baseado em troca de mensagens como por exemplo MPI \cite{Gropp1996}, voltados para aplicações mais específicas. O modelo de atores se difere deles por ser uma abstração, e não exige a existência de \textit{threads}, \textit{mailboxes}, fila de mensagens, \textit{brokers}, etc. Com isso, várias aplicações podem ser modeladas e interpretadas a partir dele, como por exemplo serviços de e-mail, podendo ser inclusive implementado diretamente no hardware. 

\subsection{Scala e Java}
A linguagem Scala começou a ser desenvolvida entre os anos de 2001 e 2004 no Laboratório de Métodos de Programação da EPFL com o intuito de prover
uma linguagem de programação de tipagem estática, puramente orientada a objetos, funcional e que facilitasse o desenvolvimento de componentes de software reutilizáveis \cite{Odersky2004}. Neste sentido o modelo de atores foi incorporado à linguagem como mecanismo de concorrência padrão por ser mais escalável que o modelo tradicional baseado em \textit{threads}, uma propriedade desejável no design da linguagem.

Na versão 2.11.0 da linguagem, a implementação do modelo de atores padrão que passa a ser utilizada é a Akka~
\footnote{http://docs.scala-lang.org/overviews/core/actors-migration-guide.html} enquanto a anterior foi marcada como \textit{deprecated}. Comparada com a implementação anterior na biblioteca padrão da linguagem Scala, a biblioteca Akka tem a vantagem de ter sido escrita não só para Scala como também para Java, o que não acontecia na primeira. É, além disso uma ferramenta mais consolidada no mercado, já utilizada por algumas empresas de tecnologia. Como ambas as implementações de Akka (em Scala e Java) só diferem em termos sintáticos das linguagens e são completamente compatíveis, aqui serão exibidos somente exemplos em Java.

Para criar um ator em Java utilizando o toolkit Akka é necessário extender a classe \textbf{UntypedActor}. Essa classe obriga o programador a implementar o método \textit{\textbf{onReceive()}}, que recebe um parâmetro do tipo \textbf{Object} e não tem retorno. Dentro desse método o ator deve tratar e responder a mensagem que recebeu. O \autoref{lst:akkaActor} exibe uma simples implementação que responde a uma \textbf{String} com uma mensagem de boas vindas.

\begin{lstlisting}[language=Java,label={lst:akkaActor}, caption={Classe que define o comportamento de um ator}]
import akka.actor.UntypedActor;

public class AtorBoasVindas extends UntypedActor {

    public void onReceive(Object msg) {
        if (msg instanceof String)     
            sender().tell("Seja bem vindo " + ((String) msg) + "\n");
    }
}
\end{lstlisting}

Todo ator ao receber uma mensagem, trata-a de forma serial, e não compartilha memória com nenhum outro ator, portanto não existem condições de corrida se as premissas do modelo forem respeitadas. Evidentemente, dois atores podem estar executando \textit{\textbf{onReceive()}} ao mesmo tempo, isso depende do tipo de escalonamento, do número de \textit{threads} disponíveis e do número de atores prontos para tratarem uma mensagem. 

Para criar um ator não se usa o construtor da classe: ele deve ser criado dentro do sistema de atores onde será encapsulado, receberá uma \textit{mailbox} e será gerenciado pelo escalonador quando precisar ser executado. A \autoref{lst:criaAtor} exibe a forma correta de criar um sistema de atores  e interagir com o ator.

\begin{lstlisting}[language=Java,label={lst:criaAtor}, caption={Programa principal que cria um sistema de atores e instancia um ator}]
import akka.actor.ActorSystem;
import akka.actor.ActorRef;

public class Exemplo1 {
    public static void main(String [] args) {
        ActorSystem system = ActorSystem.create("teste");
        ActorRef  ator = system.actorFor(Props.create(AtorBoasVindas.class),"boasVindas");
        
        ator.tell("Felipe", ActorRef.noSender());
    }
}

\end{lstlisting}

Este programa deve criar um ator (linha 7) dentro do sistema de atores fornecido pelo toolkit Akka (linha 6) e enviar uma mensagem para ele (linha 9). A chamada ao método \textit{\textbf{tell()}} é assíncrona, \textit{i.e.}, ela retorna antes da mensagem ser tratada. Todo ator possui uma \textit{mailbox} onde as mensagens são enfileiradas para serem tratadas quando o ator for escalonado. Objetos do tipo \textbf{ActorRef} encapsulam o comportamento de todos os atores do sistema e todas as mensagens passadas para um ator são passadas por cópia em detrimento de passagem por referência, garantindo que um ator não compartilhe memória com seu remetente. 

Os atores estão disponíveis sob um caminho lógico que os torna acessíveis a partir de outras instâncias do sistema (a saber, outros processos que podem, inclusive, estar executando em outra máquina na rede). O caminho é estruturado da seguinte maneira: 

\begin{lstlisting}[caption={Esquema da URI de um ator }]
 protocolo://nomeSistema[@host:porta]/user/nomeAtor 
\end{lstlisting}

Os caminhos lógicos de Akka seguem o padrão URI (Unified Resource Identification). O protocolo para acessar atores remotos deve ser "akka.tcp", caso o ator esteja na mesma máquina o protocolo é somente "akka" e não é necessário especificar o \textit{host} e a porta. Os atores criados pelo programador estão sempre abaixo do ator ``user`` que os supervisiona. Caso o ator do \autoref{lst:akkaActor} criasse um outro ator chamado "subAtor", sua URI seria "akka://teste/user/boasVindas/subAtor".

É importante também descrever a semântica de entrega das mensagens que segue a regra \textit{at-most-once}, ou seja, a mensagem será entregue, no máximo, uma vez ao ator. Isso implica em não garantia da entrega das mensagens. De acordo com a documentação do \textit{toolkit}, essa semântica foi escolhida por ser de simples implementação, pode ser usada de forma assíncrona sem manter estado das mensagens enviadas, e possui melhor desempenho comparado com outras alternativas (\textit{at-least-once} e \textit{exactly-once})\footnote{https://doc.akka.io/docs/akka/current/general/message-delivery-reliability.html}. As mensagens trocadas diretamente entre pares de atores são entregues em ordem, \textit{e.g.} se um ator A1 enviar as mensagens M1, M2 e M3 ao ator A2, se M1 chegar ao destino, chegará antes de M2; se M2 chegar ao destino chegará antes de M3. Esse último atributo é particular da implementação de Akka e não é comum ser encontrado em outras implementações.  

Por fim, o modelo de atores elimina os problemas de concorrência em sistemas não triviais permitindo ao software ganhar escalabilidade vertical e horizontal. A implementação de Akka oferece um sistema que encapsula a referência dos atores, permite transparência de localidade através dos \textit{paths}, e um sistema de comunicação que apesar de não garantir a entrega, fornece alto desempenho. Por não compartilharem estado, testar cada ator separadamente é mais fácil que testar \textit{threads} separadamente. Entretanto, para construir um sistema utilizando o paradigma de atores é necessário repensar toda a arquitetura de software.  Não basta eliminar os pontos de acesso compartilhado e substituir por troca de mensagem, é necessário pensar em um software que não tem garantias de entrega das mensagens, completamente assíncrono, e que seja tolerante a falhas.

\section{Considerações finais}
Duas técnicas de programação concorrente foram apresentadas neste capítulo, apontando vantagens e desvantagens. Programação baseada em \textit{threads} e compartilhamento de estado se mostrou ao longo da história um modo de se produzir sistemas confiáveis, apesar dos problemas de sincronização. Ferramentas e técnicas foram desenvolvidas para tratar esses problemas, entre elas semáforos, \textit{mutex} e algoritmos clássicos de sincronização. 

O modelo de atores surge na mesma época que o de \textit{threads} mas só ganha popularidade recentemente com os sistemas \textit{web} em larga escala, e a publicação do manifesto reativo \footnote{Teve sua segunda versão publicada em 2014 e é um documento já assinado por milhares de desenvolvedores que estabelece quatro características de sistemas reativos: responsividade, resiliência, elasticidade e comunicação através de mensagens assíncronas. Mais informações podem ser encontradas no link https://www.reactivemanifesto.org/pt}. Esse modelo estabelece a comunicação pela troca de mensagens assíncronas e abre mão do uso de memória compartilhada, possibilitando que os sistemas escalem vertical e horizontalmente. 

Um modelo assíncrono de programação, além de trazer ganhos de desempenho computacional à construção de softwares complexos, parece ter uma conformidade com a modelagem de alguns processos biológicos \cite{Strombom2017, Fisher2008, Wang2016}, maior que modelos síncronos e bloqueantes. Neste sentido o próximo capítulo lista alguns modelos do processo cognitivo apresentando suas bases teóricas, o seu funcionamento em alto nível, e quando possível, o ferramental tecnológico utilizado. Adiante é apresentada uma alternativa para construção de um simulador do processo cognitivo baseado no modelo de atores.