% -----------------------------------------------------------------------------
%   Arquivo: ./02-elementos-textuais/conclusao.tex
% -----------------------------------------------------------------------------



\chapter{Conclusão}
\label{chap:conclusao}

A proposta deste trabalho foi propor o modelo e a implementação de um simulador de vida artificial distribuído e assíncrono utilizando o modelo de atores como ferramenta construtitva. Como foi discutido na \autoref{chap:introducao}, há relevância em prover uma arquitetura deste perfil visto que é interessante estudar a modelagem de sistemas biológico em larga escala,  e as técnicas usadas frequentemente na literatura, como concorrência por \textit{threads}, são impeditivas para essa linha de pesquisa.

Como \autoref{chap:fundamentacaoTeorica}, o modelo de programação concorrente baseado em \textit{threads} e memória compartilhada, padece de problemas de sincronização, como condições de corrida, \mbox{\textit{deadlocks}} e \textit{livelocks}, mas é possível evitá-los com o uso correto de \textit{locks}, assumindo que há perda de performance. O modelo de atores abre mão da possibilidade de compartilhamento de memória e adota uma operação estritamente assíncrona, baseado em troca de mensagens. Essa decisão construtiva evita esses problemas de sincronização.A implementação do modelo de atores mais desenvolvida atualmente é o \textit{Akka}, e por isso foi adotada neste trabalho. 

Na área de simulação de processos biológicos, principalmente do processo cognitivo, não existem muitos projetos que pensem a modelagem em termos de sistemas distribuídos e assíncronos. No \autoref{chap:trabRelac} foram apresentados trabalhos que modelam o processo cognitivo, partindo de uma perspectiva dita biológica, mas que no fundo partem de abordagens clássicas simbolistas/conexionistas. A exceção dentre estes trabalhos é o projeto Artífice, uma arquitetura para construção de criaturas artificiais dotadas de sistema nervoso, parte da perspectiva da cognição biológica. Além do mais, este foi construído para ser um sistema multi-agentes não determinístico e assíncrono, característica não  encontrada nos outros trabalhos. Contudo o projeto padecia de problemas de escalabilidade, por ter sido construído utilizando programação concorrente baseada no compartilhamento de estado.

Tendo uma arquitetura cognitiva funcional, fundamentada num arcabouço teórico amplo da modelagem do processo cognitivo, foi proposta uma nova implementação no \autoref{chap:metodologia} baseada no modelo de atores, que permitisse a simulação em ambiente distribuído. A arquitetura proposta, nomeada DL2L tem as mesmas funcionalidades básicas da arquitetura Artífice, no que diz respeito ao processo cognitivo-emocional. Um esquema de persistência foi proposto para que seja possível armazenar e recuperar posteriormente os dados da dinâmica interna e externa das criaturas. Também foi proposto um esquema de distribuição das entidades da simulação de forma que eles sejam logicamente localizáveis em um \textit{cluster}. O modelo possui um único gargalo e ponto único de falha que é a verificação de colisões, que até aqui, é feita de forma centralizada.

O capítulo \autoref{chap:resultados} expõe os resultados dos experimentos preliminares. Estes serviram para avaliar a conformidade qualitativa da arquitetura proposta com a Artífice, tal como a escalabilidade vertical e horizontal do sistema. Os resultados mostraram que, a princípio ambas as arquiteturas são qualitativamente comparáveis. Contudo o projeto DL2L ainda carece de refinamentos, isso pois a tarefa de detecção de colisões parece ficar sobrecarregada quando o número de criaturas aumenta numa taxa pequena. 

\section{Principais contribuições}


Este trabalho deixa como principal contribuição uma arquitetura cognitiva multi-agente para simulação de criaturas artificiais, que opera como um sistema assíncrono distribuído. A arquitetura DL2L é qualitativamente comparável com a arquitetura Artífice, guardadas as diferenças entre os parâmetros intrínsecos e extrínsecos a criatura artificial. Do ponto de vista da engenharia de software, a primeira tem um código muito mais simples, com menos parâmetros para serem calibrados do que a segunda. Isso por que no Artífice havia uma \textit{thread} para cada componente, e era necessário calibrar parâmetros para o ambiente de execução de todas essas \textit{threads}. Na arquitetura DL2L o ambiente de execução está encapsulado pelo \textit{toolkit} Akka. 

Apesar da arquitetura ainda não tornar possível o estudo de simulações de forrageamento em larga escala por sofrer de um problema de único ponto de gargalo/falha, espera-se que resolvido este problema seja possível escalar o sistema em um \textit{cluster}. 


\section{Trabalhos futuros}

Como trabalhos futuros propõe-se estudar outras maneiras de verificar a colisão em um ambiente distribuído de modo a evitar o aumento  da latência a medida que o sistema cresce. Possibilidades que podem ser exploradas a resolver esse problema: checar colisões de mais entidades ao mesmo tempo e responder todas de uma vez, evitando o aumento da quantidade de mensagens enviadas; permitir ao ator que verifica as colisões de criar outros atores que possam assumir a tarefa, aproveitando o paralelismo da máquina; alterar o esquema de alocação das entidades entre os \textit{holders}, de modo que seja possível, dada a posição da criatura, descobrir em tempo constante quais \textit{holders} tem entidades próxima a essa região. É necessário também aprimorar os mecanismos de aprendizagem, adicionando memórias de curto, médio e longo prazo e comparar os resultados com os já vistos na literatura. 


%\section{Considerações Finais}
%\label{sec:consideracoesFinais}

%As derradeiras palavras para encerramento do seu trabalho acadêmico.



% -----------------------------------------------------------------------------
%   OBS: a norma ABNT estabelece que em qualquer tipo de trabalho acadêmico monográfico
%   deve haver um capítulo de conclusão
% -----------------------------------------------------------------------------
